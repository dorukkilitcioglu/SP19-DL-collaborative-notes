\chapter{Visualizing 2D interpolation}
% Authors: Doruk Kilitcioglu, Pedro Manuel Herrero Vidal, Yves Greatti
% Lecture date: 2/25/2019

As with the previous parts, we have three curves defined by functions:
\[X_c(t) = t
\begin{pmatrix}
    \sin( \frac{2 \pi}{C}(2t + c -1)) \\
    \cos( \frac{2 \pi}{C}(2t + c -1))
\end{pmatrix}
 \]
\[ 0 \leq t \leq 1 \quad c = 1,..., C\]

\begin{figure}[ht]
    \centering
    \includegraphics[width=200pt]{labs/02/images/spiral1.png}
    \caption{3 not linearly separable parametric curves}
    \label{fig:spiral2}
\end{figure}

We use a 3 layer net to visualize it. The input is 2D, the first hidden layer brings it up to 100D, after which there is a ReLU non-linearity.
Compared to last time, we have a wider layer, which allows for more opportunities in representing the data.
We then go back down to 2D again, which is useful for visualization, and then we can bring it back up to 3D for the final classification.
Since there isn't any non-linearity, the last two layers are equivalent to having a single layer that goes from the ReLU to the classification layer.

In order to visualize each transformation, we do a linear interpolation from the input to the output at the 2D hidden layer using

\[(1-\alpha)x^{(i)} + \alpha \phi(x^{(i)}), 0\le \alpha \le 1
 \]